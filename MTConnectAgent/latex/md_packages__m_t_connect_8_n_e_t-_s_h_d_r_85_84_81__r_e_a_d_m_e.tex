Classes to handle the S\+H\+DR Agent Adapter Protocol associated with the M\+T\+Connect Standard.

\subsection*{Overview}

The Shdr\+Adapter classes handle the T\+CP connection to the Agent\+:


\begin{DoxyItemize}
\item \href{https://github.com/TrakHound/MTConnect.NET/blob/master/src/MTConnect.NET-SHDR/Adapters/Shdr/ShdrAdapter.cs}{\tt Shdr\+Adapter} \+: Sends the most recent values On-\/\+Demand using the Send\+Changed() method. This is used when full control of the communication is needed.
\item \href{https://github.com/TrakHound/MTConnect.NET/blob/master/src/MTConnect.NET-SHDR/Adapters/Shdr/ShdrIntervalAdapter.cs}{\tt Shdr\+Interval\+Adapter} \+: Sends the most recent values at the specified Interval. This is used when a set interval is adequate and the most recent value is all that is needed
\item \href{https://github.com/TrakHound/MTConnect.NET/blob/master/src/MTConnect.NET-SHDR/Adapters/Shdr/ShdrQueueAdapter.cs}{\tt Shdr\+Queue\+Adapter} \+: Queues all values that are sent from the P\+LC and sends them all on demand using the Send\+Buffer() method. This is used when all values are needed and full control of the communication is needed.
\item \href{https://github.com/TrakHound/MTConnect.NET/blob/master/src/MTConnect.NET-SHDR/Adapters/Shdr/ShdrIntervalQueueAdapter.cs}{\tt Shdr\+Interval\+Queue\+Adapter} \+: Queues all values that are sent from the P\+LC and sends any queued values at the specified Interval. This is used when all values are needed but an interval is adequate.
\end{DoxyItemize}

S\+H\+DR conversion is handled in each individual class\+:
\begin{DoxyItemize}
\item \href{Shdr/ShdrDataItem.cs}{\tt Shdr\+Data\+Item} \+: Handles converting Events and/or Samples with a Representation of V\+A\+L\+UE to the appropriate S\+H\+DR format.
\item \href{Shdr/ShdrCondition.cs}{\tt Shdr\+Condition} \+: Handles converting Conditions to the appropriate S\+H\+DR format
\item \href{Shdr/ShdrTimeSeries.cs}{\tt Shdr\+Time\+Series} \+: Handles converting Samples with a Representation of T\+I\+M\+E\+\_\+\+S\+E\+R\+I\+ES to the appropriate S\+H\+DR format
\item \href{Shdr/ShdrDataSet.cs}{\tt Shdr\+Data\+Set} \+: Handles converting Events and/or Samples with a Representation of D\+A\+T\+A\+\_\+\+S\+ET to the appropriate S\+H\+DR format
\item \href{Shdr/ShdrTable.cs}{\tt Shdr\+Table} \+: Handles converting Events and/or Samples with a Representation of T\+A\+B\+LE to the appropriate S\+H\+DR format
\item \href{Shdr/ShdrAsset.cs}{\tt Shdr\+Asset} \+: Handles converting Assets to the appropriate S\+H\+DR format
\end{DoxyItemize}

The \href{Adapters/Shdr/ShdrAdapterClient.cs}{\tt Shdr\+Adapter\+Client} class handles the T\+CP connection to read from the Adapter and add data to an I\+M\+T\+Connect\+Agent class.

\subsection*{Usage}

There are several different ways to setup and add data to the Shdr\+Adapter

\subsubsection*{Shdr\+Adapter}

The example below creates a new Shdr\+Adapter. The added Data\+Item will be sent using the \char`\"{}adapter.\+Send\+Changed()\char`\"{} method 
\begin{DoxyCode}
\{c#\}
using MTConnect.Adapters.Shdr;

ShdrAdapter adapter = new ShdrAdapter();
adapter.Start();

adapter.AddDataItem("L2estop", "ARMED");
adapter.SendChanged();
\end{DoxyCode}
 \#\#\#\# S\+H\+DR Output 
\begin{DoxyCode}
2023-01-26T16:48:17.0206852Z|L2estop|ARMED
\end{DoxyCode}


\subsubsection*{Shdr\+Adapter (specific Port)}

The example below creates a new Shdr\+Adapter on the 7980 T\+CP Port. 
\begin{DoxyCode}
\{c#\}
using MTConnect.Adapters.Shdr;

ShdrAdapter adapter = new ShdrAdapter(7980);
\end{DoxyCode}


\subsubsection*{Shdr\+Adapter (specific Device)}

The example below creates a new Shdr\+Adapter for the \char`\"{}\+O\+K\+U\+M\+A-\/\+Lathe\char`\"{} device. The added Data\+Item will be sent using the \char`\"{}adapter.\+Send\+Changed()\char`\"{} method 
\begin{DoxyCode}
\{c#\}
using MTConnect.Adapters.Shdr;

ShdrAdapter adapter = new ShdrAdapter("OKUMA-Lathe", 7980);
adapter.Start();

adapter.AddDataItem("L2estop", "ARMED");
adapter.SendChanged();
\end{DoxyCode}
 \#\#\#\# S\+H\+DR Output 
\begin{DoxyCode}
2023-01-26T20:54:34.1694626Z|OKUMA-Lathe:L2estop|ARMED
\end{DoxyCode}


\subsubsection*{Shdr\+Adapter (specific Device and Port)}

The example below creates a new Shdr\+Adapter for the \char`\"{}\+O\+K\+U\+M\+A-\/\+Lathe\char`\"{} device on the 7980 T\+CP Port. 
\begin{DoxyCode}
\{c#\}
using MTConnect.Adapters.Shdr;

ShdrAdapter adapter = new ShdrAdapter("OKUMA-Lathe", 7980);
\end{DoxyCode}


\subsubsection*{Shdr\+Interval\+Adapter}

The example below creates a new Shdr\+Interval\+Adapter and sets the Interval to 500ms. Only the most recent Data\+Item with the ID = \char`\"{}\+L2estop\char`\"{} will be sent after 500ms has elapsed. 
\begin{DoxyCode}
\{c#\}
using MTConnect.Adapters.Shdr;

ShdrIntervalAdapter adapter = new ShdrIntervalAdapter(interval: 500);
adapter.Start();

adapter.AddDataItem("L2estop", "ARMED");
adapter.AddDataItem("L2estop", "TRIGGERED");
adapter.AddDataItem("L2estop", "ARMED");
adapter.AddDataItem("L2estop", "TRIGGERED");
\end{DoxyCode}
 \#\#\#\# S\+H\+DR Output 
\begin{DoxyCode}
2023-01-26T17:22:10.3128693Z|L2estop|TRIGGERED
\end{DoxyCode}


\subsubsection*{Shdr\+Queue\+Adapter}

The example below creates a new Shdr\+Queue\+Adapter. All of the added Data\+Items will be sent using the \char`\"{}adapter.\+Send\+Buffer()\char`\"{} method 
\begin{DoxyCode}
\{c#\}
using MTConnect.Adapters.Shdr;

ShdrQueueAdapter adapter = new ShdrQueueAdapter();
adapter.Start();

adapter.AddDataItem("L2estop", "ARMED");
adapter.AddDataItem("L2estop", "TRIGGERED");
adapter.AddDataItem("L2estop", "ARMED");
adapter.AddDataItem("L2estop", "TRIGGERED");
adapter.SendBuffer();
\end{DoxyCode}
 \#\#\#\# S\+H\+DR Output 
\begin{DoxyCode}
2023-01-26T17:22:10.2882160Z|L2estop|ARMED
2023-01-26T17:22:10.3125127Z|L2estop|TRIGGERED
2023-01-26T17:22:10.3128389Z|L2estop|ARMED
2023-01-26T17:22:10.3128693Z|L2estop|TRIGGERED
\end{DoxyCode}


\subsubsection*{Shdr\+Interval\+Queue\+Adapter}

The example below creates a new Shdr\+Interval\+Queue\+Adapter and sets the Interval to 500ms. All of the added Data\+Items will be sent after 500ms has elapsed. 
\begin{DoxyCode}
\{c#\}
using MTConnect.Adapters.Shdr;

ShdrIntervalQueueAdapter adapter = new ShdrIntervalQueueAdapter(interval: 500);
adapter.Start();

adapter.AddDataItem("L2estop", "ARMED");
adapter.AddDataItem("L2estop", "TRIGGERED");
adapter.AddDataItem("L2estop", "ARMED");
adapter.AddDataItem("L2estop", "TRIGGERED");
\end{DoxyCode}
 \#\#\#\# S\+H\+DR Output 
\begin{DoxyCode}
2023-01-26T17:22:10.2882160Z|L2estop|ARMED
2023-01-26T17:22:10.3125127Z|L2estop|TRIGGERED
2023-01-26T17:22:10.3128389Z|L2estop|ARMED
2023-01-26T17:22:10.3128693Z|L2estop|TRIGGERED
\end{DoxyCode}


\subsection*{Configuration}


\begin{DoxyItemize}
\item {\ttfamily Id} -\/ Get a unique identifier for the Adapter
\item {\ttfamily Device\+Key} -\/ The Name or U\+U\+ID of the Device to create a connection for
\item {\ttfamily Port} -\/ The T\+CP Port used for communication
\item {\ttfamily Heartbeat} -\/ The heartbeat used to maintain a connection between the Adapter and the Agent
\item {\ttfamily Connection\+Timeout} -\/ The amount of time (in milliseconds) to allow for a connection attempt to the Agent
\item {\ttfamily Reconnect\+Interval} -\/ The amount of time (in milliseconds) between adapter reconnection attempts
\item {\ttfamily Multiline\+Assets} -\/ Use multiline Assets
\item {\ttfamily Multiline\+Devices} -\/ Sets the default for Converting Units when adding Observations
\item {\ttfamily Filter\+Duplicates} -\/ Determines whether to filter out duplicate data
\item {\ttfamily Output\+Timestamps} -\/ Determines whether to output Timestamps for each S\+H\+DR line
\end{DoxyItemize}

\subsubsection*{Output\+Timestamps}

Timestamps may be output in the S\+H\+DR protocol in order to pass the timestamp from the Adapter instead of allowing the Agent to apply a timestamp.

{\bfseries Output\+Timestamp} = true 
\begin{DoxyCode}
2023-01-27T03:47:53.4477372Z|L2estop|ARMED
\end{DoxyCode}
 {\bfseries Output\+Timestamp} = false 
\begin{DoxyCode}
L2estop|ARMED
\end{DoxyCode}


\subsection*{Sending Data}

\#\#\# Add Data\+Items Individually 
\begin{DoxyCode}
\{c#\}
// DataItemId and CDATA
adapter.AddDataItem("L2estop", "ARMED");

// DataItemId and CDATA, and Timestamp
adapter.AddDataItem("L2estop", "ARMED", DateTime.UtcNow);

// DataItemId and CDATA as ShdrDataItem
adapter.AddDataItem(new ShdrDataItem("L2estop", "ARMED"));

// DataItemId, CDATA, and Timestamp as ShdrDataItem
adapter.AddDataItem(new ShdrDataItem("L2estop", "ARMED", DateTime.UtcNow));
\end{DoxyCode}
 \begin{quote}
If no timestamp is given then the timestamp will be set when the Data\+Item is sent to the Agent. To insure that all dataitems have the same timestamp, it can be set explicitly. \end{quote}


\#\#\# Add List of Data\+Items 
\begin{DoxyCode}
\{c#\}
var ts = DateTime.UtcNow;
var dataItems = new List<ShdrDataItem>();

dataItems.Add(new ShdrDataItem("L2p1execution", "READY", ts));
dataItems.Add(new ShdrDataItem("L2p1Fovr", 100, ts));
dataItems.Add(new ShdrDataItem("L2p1partcount", 15, ts));
dataItems.Add(new ShdrDataItem("L2p1Fact", 250, ts));

adapter.AddDataItems(dataItems);
\end{DoxyCode}
 \#\#\#\# Output 
\begin{DoxyCode}
2023-01-26T20:50:53.6161001Z|L2p1execution|READY|L2p1Fovr|100|L2p1partcount|15|L2p1Fact|250
\end{DoxyCode}


\#\#\# Store Data\+Items in Variables 
\begin{DoxyCode}
\{c#\}
ShdrDataItem executionDataItem = new ShdrDataItem("L2p1execution");
ShdrDataItem feedrateOverrideDataItem = new ShdrDataItem("L2p1Fovr");
ShdrDataItem partCountDataItem = new ShdrDataItem("L2p1partcount");
ShdrDataItem feedrateActualDataItem = new ShdrDataItem("L2p1Fact");

executionDataItem.Value = "READY";
feedrateOverrideDataItem.Value = 100;
partCountDataItem.Value = 15;
feedrateActualDataItem.Value = 250;

adapter.AddDataItem(executionDataItem);
adapter.AddDataItem(feedrateOverrideDataItem);
adapter.AddDataItem(partCountDataItem);
adapter.AddDataItem(feedrateActualDataItem);
\end{DoxyCode}


\subsubsection*{Send Data\+Item Manually}

Data\+Items can be added to the Adapter and immediately send to the Agent using the \char`\"{}\+Send\+Data\+Item()\char`\"{} method 
\begin{DoxyCode}
\{c#\}
adapter.SendDataItem("L2estop", "ARMED");
\end{DoxyCode}


\subsection*{Conditions}

The Shdr\+Condition class is used to send M\+T\+Connect Condition data using the 5 primary values Level, Native\+Code, Native\+Severity, Qualifier, and Message (referred to as Result in the M\+T\+Connect Standard). A Condition is made up of 1 or more Fault\+States. Each Fault\+State is represented by the Shdr\+Fault\+State class.

\#\#\#\# Add a Fault Condition 
\begin{DoxyCode}
\{c#\}
ShdrCondition condition = new ShdrCondition("L2p1system", ConditionLevel.FAULT);
adapter.AddCondition(condition);
\end{DoxyCode}


A Condition can also be added by using the built-\/in functions\+:

\#\#\#\# Set the Condition to Normal 
\begin{DoxyCode}
\{c#\}
ShdrCondition condition = new ShdrCondition("L2p1system");
condition.Normal();

adapter.AddCondition(condition);
\end{DoxyCode}


\#\#\#\# Set the Condition to Warning 
\begin{DoxyCode}
\{c#\}
ShdrCondition condition = new ShdrCondition("L2p1system");
condition.Warning("Not Found", "404", "100", ConditionQualifier.LOW);

adapter.AddCondition(condition);
\end{DoxyCode}


\#\#\#\# Set the Condition to Fault 
\begin{DoxyCode}
\{c#\}
ShdrCondition condition = new ShdrCondition("L2p1system");
condition.Fault("Internal Error", "500", "10254", ConditionQualifier.HIGH);

adapter.AddCondition(condition);
\end{DoxyCode}


\#\#\#\# Set multiple Fault\+States 
\begin{DoxyCode}
\{c#\}
ShdrCondition condition = new ShdrCondition("L2p1coolant");
condition.AddWarning("Coolant Level Low", "47321", qualifier: ConditionQualifier.LOW);
condition.AddWarning("Coolant Temperature High", "98712", qualifier: ConditionQualifier.HIGH);

adapter.AddCondition(condition);
\end{DoxyCode}


\#\#\#\# Set the Condition to Unavailable 
\begin{DoxyCode}
\{c#\}
ShdrCondition condition = new ShdrCondition("L2p1system");
condition.Unavailable();

adapter.AddCondition(condition);
\end{DoxyCode}


\#\#\#\# Add Fault\+States Individually 
\begin{DoxyCode}
\{c#\}
ShdrCondition condition = new ShdrCondition("L2p1system");

ShdrFaultState faultState = new ShdrFaultState();
faultState.NativeCode = "404";
faultState.NativeSeverity = "100";
faultState.Qualifier = "LOW";
faultState.Message = "Testing from new adapter";

condition.AddFaultState(faultState);
adapter.AddCondition(condition);
\end{DoxyCode}


\subsubsection*{Send Condition Manually}

Conditions can be added to the Adapter and immediately send to the Agent using the \char`\"{}\+Send\+Condition()\char`\"{} method 
\begin{DoxyCode}
\{c#\}
adapter.SendCondition(condition);
\end{DoxyCode}


\#\# Time\+Series 
\begin{DoxyCode}
\{c#\}
List<double> samples = new List<double>();
samples.Add(12);
samples.Add(15);
samples.Add(14);
samples.Add(18);
samples.Add(25);
samples.Add(30);

ShdrTimeSeries timeSeries = new ShdrTimeSeries("L2p1Sensor", samples, 100);

adapter.AddTimeSeries(timeSeries);
\end{DoxyCode}
 \#\#\# Output 
\begin{DoxyCode}
2023-01-26T20:39:28.1540686Z|L2p1Sensor|6|100|12 15 14 18 25 30
\end{DoxyCode}


\subsubsection*{Send Time\+Series Manually}

Time\+Series can be added to the Adapter and immediately send to the Agent using the \char`\"{}\+Send\+Time\+Series()\char`\"{} method 
\begin{DoxyCode}
\{c#\}
adapter.SendTimeSeries(timeSeries);
\end{DoxyCode}


\#\# Data\+Sets 
\begin{DoxyCode}
\{c#\}
List<DataSetEntry> dataSetEntries = new List<DataSetEntry>();
dataSetEntries.Add(new DataSetEntry("V1", 5));
dataSetEntries.Add(new DataSetEntry("V2", 205));

ShdrDataSet dataSet = new ShdrDataSet("L2p1Variables", dataSetEntries);

adapter.AddDataSet(dataSet);
\end{DoxyCode}
 \#\#\# Output 
\begin{DoxyCode}
2023-01-26T20:40:30.6718334Z|L2p1Variables|V1=5 V2=205
\end{DoxyCode}


\subsubsection*{Send Data\+Sets Manually}

Data\+Sets can be added to the Adapter and immediately send to the Agent using the \char`\"{}\+Send\+Data\+Sets()\char`\"{} method 
\begin{DoxyCode}
\{c#\}
adapter.SendDataSets(dataSet);
\end{DoxyCode}


\#\# Tables 
\begin{DoxyCode}
\{c#\}
List<TableEntry> tableEntries = new List<TableEntry>();

// Tool 1
List<TableCell> t1Cells = new List<TableCell>();
t1Cells.Add(new TableCell("LENGTH", 7.123));
t1Cells.Add(new TableCell("DIAMETER", 0.494));
t1Cells.Add(new TableCell("REMAINING\_LIFE", 35));
tableEntries.Add(new TableEntry("T1", t1Cells));

// Tool 2
List<TableCell> t2Cells = new List<TableCell>();
t2Cells.Add(new TableCell("LENGTH", 10.456));
t2Cells.Add(new TableCell("DIAMETER", 0.125));
t2Cells.Add(new TableCell("REMAINING\_LIFE", 100));
tableEntries.Add(new TableEntry("T2", t2Cells));

// Tool 3
List<TableCell> t3Cells = new List<TableCell>();
t3Cells.Add(new TableCell("LENGTH", 6.251));
t3Cells.Add(new TableCell("DIAMETER", 1.249));
t3Cells.Add(new TableCell("REMAINING\_LIFE", 93));
tableEntries.Add(new TableEntry("T3", t3Cells));

ShdrTable table = new ShdrTable("L2p1ToolTable", tableEntries);

adapter.AddTable(table);
\end{DoxyCode}
 \#\#\# Output 
\begin{DoxyCode}
2023-01-26T20:40:55.8702675Z|L2p1ToolTable|T1=\{LENGTH=7.123 DIAMETER=0.494 TOOL\_LIFE=0.35\}
       T2=\{LENGTH=10.456 DIAMETER=0.125 TOOL\_LIFE=1\} T3=\{LENGTH=6.251 DIAMETER=1.249 TOOL\_LIFE=0.93\}
\end{DoxyCode}


\subsubsection*{Send Tables Manually}

Tables can be added to the Adapter and immediately send to the Agent using the \char`\"{}\+Send\+Tables()\char`\"{} method 
\begin{DoxyCode}
\{c#\}
adapter.SendTables(table);
\end{DoxyCode}


\subsection*{Assets}

M\+T\+Connect Assets are sent by first defining the Asset using the appropriate class (Cutting\+Tool\+Asset, File\+Asset, etc.) then using the \char`\"{}\+Add\+Asset()\char`\"{}.

\#\#\# Cutting\+Tool Asset 
\begin{DoxyCode}
\{c#\}
using MTConnect.Assets.CuttingTools;
using MTConnect.Assets.CuttingTools.Measurements;

var tool = new CuttingToolAsset();
tool.AssetId = "5.12";
tool.ToolId = "12";
tool.CuttingToolLifeCycle = new CuttingToolLifeCycle
\{
    Location = new Location \{ Type = LocationType.SPINDLE \},
    ProgramToolNumber = "12",
    ProgramToolGroup = "5"
\};
tool.CuttingToolLifeCycle.Measurements.Add(new FunctionalLengthMeasurement(7.6543));
tool.CuttingToolLifeCycle.Measurements.Add(new CuttingDiameterMaxMeasurement(0.375));
tool.CuttingToolLifeCycle.CuttingItems.Add(new CuttingItem
\{
    ItemId = "12.1",
    Locus = CuttingItemLocas.FLUTE.ToString()
\});
tool.CuttingToolLifeCycle.CutterStatus.Add(CutterStatus.AVAILABLE);
tool.CuttingToolLifeCycle.CutterStatus.Add(CutterStatus.NEW);
tool.CuttingToolLifeCycle.CutterStatus.Add(CutterStatus.MEASURED);
tool.DateTime = DateTime.Now;

adapter.AddAsset(tool);
\end{DoxyCode}
 \#\#\#\# Output (with Multiline\+Assets = true) 
\begin{DoxyCode}
2023-01-26T17:56:59.9694353Z|@ASSET@|5.12|CuttingTool|--multiline--W5XZBJ2QZV
<CuttingTool assetId="5.12" timestamp="2023-01-26T12:56:59.4778578-05:00" toolId="12">
  <CuttingToolLifeCycle>
    <CutterStatus>
      <Status>AVAILABLE</Status>
      <Status>NEW</Status>
      <Status>MEASURED</Status>
    </CutterStatus>
    <Location type="SPINDLE">0</Location>
    <ProgramToolGroup>5</ProgramToolGroup>
    <ProgramToolNumber>12</ProgramToolNumber>
    <Measurements>
      <FunctionalLength units="MILLIMETER" code="LF">7.6543</FunctionalLength>
      <CuttingDiameterMax units="MILLIMETER" code="DC">0.375</CuttingDiameterMax>
    </Measurements>
    <CuttingItems count="1">
      <CuttingItem itemId="12.1">
        <Locus>FLUTE</Locus>
      </CuttingItem>
    </CuttingItems>
  </CuttingToolLifeCycle>
</CuttingTool>
--multiline--W5XZBJ2QZV
\end{DoxyCode}


\#\#\# File Asset 
\begin{DoxyCode}
\{c#\}
using MTConnect.Assets.Files;

var file = new FileAsset();
file.DateTime = DateTime.UtcNow;
file.AssetId = "file.test";
file.Size = 12346;
file.VersionId = "test-v1";
file.State = FileState.PRODUCTION;
file.Name = "file-123.txt";
file.MediaType = "text/plain";
file.ApplicationCategory = ApplicationCategory.DEVICE;
file.ApplicationType = ApplicationType.DATA;
file.FileLocation = new FileLocation(@"C:\(\backslash\)temp\(\backslash\)file-123.txt");
file.CreationTime = DateTime.Now;

adapter.AddAsset(file);
\end{DoxyCode}
 \#\#\#\# Output (with Multiline\+Assets = true) 
\begin{DoxyCode}
2023-01-26T18:01:50.3085245Z|@ASSET@|file.test|File|--multiline--6UH71Y7IYW
<File assetId="file.test" timestamp="2023-01-26T18:01:49.9929867Z" name="file-123.txt"
       mediaType="text/plain" applicationCategory="DEVICE" applicationType="DATA" size="12346" versionId="test-v1" state="PRODUCTION">
  <FileLocation href="C:\(\backslash\)temp\(\backslash\)file-123.txt" />
  <CreationTime>2023-01-26T13:01:49.9939538-05:00</CreationTime>
</File>
--multiline--6UH71Y7IYW
\end{DoxyCode}


\#\#\# Add Multiple Assets 
\begin{DoxyCode}
\{c#\}
using MTConnect.Assets.CuttingTools;
using MTConnect.Assets.CuttingTools.Measurements;
using MTConnect.Assets.Files;

var assets = new List<IAsset>();


// Add the Cutting Tool Asset to the "assets" list variable
var tool = new CuttingToolAsset();
tool.AssetId = "5.12";
tool.ToolId = "12";
tool.CuttingToolLifeCycle = new CuttingToolLifeCycle
\{
    Location = new Location \{ Type = LocationType.SPINDLE \},
    ProgramToolNumber = "12",
    ProgramToolGroup = "5"
\};
tool.CuttingToolLifeCycle.Measurements.Add(new FunctionalLengthMeasurement(7.6543));
tool.CuttingToolLifeCycle.Measurements.Add(new CuttingDiameterMaxMeasurement(0.375));
tool.CuttingToolLifeCycle.CuttingItems.Add(new CuttingItem
\{
    ItemId = "12.1",
    Locus = CuttingItemLocas.FLUTE.ToString()
\});
tool.CuttingToolLifeCycle.CutterStatus.Add(CutterStatus.AVAILABLE);
tool.CuttingToolLifeCycle.CutterStatus.Add(CutterStatus.NEW);
tool.CuttingToolLifeCycle.CutterStatus.Add(CutterStatus.MEASURED);
tool.DateTime = DateTime.Now;

assets.Add(tool);


// Add the File Asset to the "assets" list variable
var file = new FileAsset();
file.DateTime = DateTime.UtcNow;
file.AssetId = "file.test";
file.Size = 12346;
file.VersionId = "test-v1";
file.State = FileState.PRODUCTION;
file.Name = "file-123.txt";
file.MediaType = "text/plain";
file.ApplicationCategory = ApplicationCategory.DEVICE;
file.ApplicationType = ApplicationType.DATA;
file.FileLocation = new FileLocation(@"C:\(\backslash\)temp\(\backslash\)file-123.txt");
file.CreationTime = DateTime.Now;

assets.Add(file);

adapter.AddAssets(assets);
\end{DoxyCode}


\subsubsection*{Send Asset Manually}

Assets can be added to the Adapter and immediately send to the Agent using the \char`\"{}\+Send\+Asset()\char`\"{} method 
\begin{DoxyCode}
\{c#\}
using MTConnect.Assets.Files;

var file = new FileAsset();
// ..
// ..
// ..

adapter.SendAsset(file);
\end{DoxyCode}


\#\#\# Remove Individual Asset by Asset\+Id 
\begin{DoxyCode}
\{c#\}
adapter.RemoveAsset("file.test");
\end{DoxyCode}
 \#\#\#\# Output 
\begin{DoxyCode}
2023-01-26T18:21:57.8208518Z|@REMOVE\_ASSET@|file.test
\end{DoxyCode}


\#\#\# Remove All Assets of a specified Type 
\begin{DoxyCode}
\{c#\}
adapter.RemoveAllAssets("File");
\end{DoxyCode}
 \#\#\#\# Output 
\begin{DoxyCode}
2023-01-26T18:31:23.6664032Z|@REMOVE\_ALL\_ASSETS@|File
\end{DoxyCode}


\subsection*{S\+H\+DR Conversion}

Conversion to and from an S\+H\+DR message is done through methods in the Shdr\+Data\+Item, Shdr\+Condition, Shdr\+Time\+Series, Shdr\+Data\+Set, and Shdr\+Table classes. Each class overrides the base \char`\"{}\+To\+String()\char`\"{} method and also contains a \char`\"{}\+From\+String()\char`\"{} method. The \char`\"{}\+To\+String()\char`\"{} method creates an S\+H\+DR compatible string from the object and the \char`\"{}\+From\+String()\char`\"{} creates an object from an S\+H\+DR string.

\#\#\# Shdr\+Data\+Item.\+To\+String() 
\begin{DoxyCode}
\{c#\}
ShdrDataItem availableDataItem = new ShdrDataItem("L2avail", Availability.AVAILABLE, DateTime.UtcNow);
Console.WriteLine(availableDataItem);
\end{DoxyCode}
 \begin{quote}
2022-\/02-\/01\+T13\+:53\+:03.\+6940000Z$\vert$\+L2avail$\vert$\+A\+V\+A\+I\+L\+A\+B\+LE \end{quote}


\#\#\# Shdr\+Condition.\+To\+String() 
\begin{DoxyCode}
\{c#\}
ShdrCondition condition = new ShdrCondition("L2p1system", ConditionLevel.FAULT, DateTime.UtcNow);
condition.NativeCode = "404";
condition.NativeSeverity = "100";
condition.Qualifier = "LOW";
condition.Message = "Testing from new adapter";
Console.WriteLine(condition);
\end{DoxyCode}
 \begin{quote}
2022-\/02-\/01\+T13\+:55\+:11.\+8460000Z$\vert$\+L2p1system$\vert$\+F\+A\+U\+L\+T$\vert$404$\vert$100$\vert$\+L\+O\+W$\vert$\+Testing from new adapter \end{quote}


\#\#\# Shdr\+Time\+Series.\+To\+String() 
\begin{DoxyCode}
\{c#\}
List<double> samples = new List<double>();
samples.Add(12);
samples.Add(15);
samples.Add(14);
samples.Add(18);
samples.Add(25);
samples.Add(30);

ShdrTimeSeries timeSeries = new ShdrTimeSeries("L2p1Sensor", samples, 100, DateTime.UtcNow);
Console.WriteLine(timeSeries);
\end{DoxyCode}
 \begin{quote}
2022-\/02-\/01\+T13\+:56\+:58.\+7700000Z$\vert$\+L2p1\+Sensor$\vert$6$\vert$100$\vert$12 15 14 18 25 30 \end{quote}


\#\#\# Shdr\+Data\+Set.\+To\+String() 
\begin{DoxyCode}
\{c#\}
List<DataSetEntry> dataSetEntries = new List<DataSetEntry>();
dataSetEntries.Add(new DataSetEntry("V1", 5));
dataSetEntries.Add(new DataSetEntry("V2", 205));

ShdrDataSet dataSet = new ShdrDataSet("L2p1Variables", dataSetEntries, DateTime.UtcNow);
Console.WriteLine(dataSet);
\end{DoxyCode}
 \begin{quote}
2022-\/02-\/01\+T13\+:58\+:31.\+8150000Z$\vert$\+L2p1\+Variables$\vert$\+V1=5 V2=205 \end{quote}


\#\#\# Shdr\+Table.\+To\+String() 
\begin{DoxyCode}
\{c#\}
List<TableEntry> tableEntries = new List<TableEntry>();

// Tool 1
List<TableCell> t1Cells = new List<TableCell>();
t1Cells.Add(new TableCell("LENGTH", 7.123));
t1Cells.Add(new TableCell("DIAMETER", 0.494));
t1Cells.Add(new TableCell("TOOL\_LIFE", 0.35));
tableEntries.Add(new TableEntry("T1", t1Cells));

// Tool 2
List<TableCell> t2Cells = new List<TableCell>();
t2Cells.Add(new TableCell("LENGTH", 10.456));
t2Cells.Add(new TableCell("DIAMETER", 0.125));
t2Cells.Add(new TableCell("TOOL\_LIFE", 1));
tableEntries.Add(new TableEntry("T2", t2Cells));

// Tool 3
List<TableCell> t3Cells = new List<TableCell>();
t3Cells.Add(new TableCell("LENGTH", 6.251));
t3Cells.Add(new TableCell("DIAMETER", 1.249));
t3Cells.Add(new TableCell("TOOL\_LIFE", 0.93));
tableEntries.Add(new TableEntry("T3", t3Cells));

ShdrTable table = new ShdrTable("L2p1ToolTable", tableEntries, DateTime.UtcNow);
Console.WriteLine(table);
\end{DoxyCode}
 \begin{quote}
2022-\/02-\/01\+T13\+:59\+:47.\+5980000Z$\vert$\+L2p1\+Tool\+Table$\vert$\+T1=\{L\+E\+N\+G\+TH=7.\+123 D\+I\+A\+M\+E\+T\+ER=0.\+494 T\+O\+O\+L\+\_\+\+L\+I\+FE=0.\+35\} T2=\{L\+E\+N\+G\+TH=10.\+456 D\+I\+A\+M\+E\+T\+ER=0.\+125 T\+O\+O\+L\+\_\+\+L\+I\+FE=1\} T3=\{L\+E\+N\+G\+TH=6.\+251 D\+I\+A\+M\+E\+T\+ER=1.\+249 T\+O\+O\+L\+\_\+\+L\+I\+FE=0.\+93\}\end{quote}
